%--------------------
% Packages
% -------------------
\documentclass[11pt,english]{article}
\usepackage{amsfonts}
\usepackage[left=2.5cm,top=2cm,right=2.5cm,bottom=3cm,bindingoffset=0cm]{geometry}
\usepackage{amsmath, amsthm, amssymb}
\usepackage{tikz}
\usetikzlibrary{calc}
\usetikzlibrary{decorations.pathreplacing,calligraphy}
\usepackage{fancyhdr}
%\usepackage{currfile}
\usepackage{nicefrac}
\usepackage{cite}
\usepackage{graphicx}
\usepackage{caption}
\usepackage{longtable}
\usepackage{rotating}
\usepackage{lscape}
\usepackage{booktabs}
\usepackage{float}
\usepackage{placeins}
\usepackage{setspace}
\usepackage[font=itshape]{quoting}
\onehalfspacing
\usepackage{mathrsfs}
\usepackage{tcolorbox}
\usepackage{xcolor}
\usepackage{subcaption}
\usepackage{float}
\usepackage[multiple]{footmisc}
\usepackage[T1]{fontenc}
\usepackage[sc]{mathpazo}
\usepackage{listings}
\usepackage{longtable}
\definecolor{cmured}{RGB}{175,30,45}
\definecolor{macroblue}{RGB}{56,108,176}
\usepackage[format=plain, 
labelfont=bf,
textfont=]{caption}
\usepackage[colorlinks=true,citecolor=macroblue,linkcolor=macroblue,urlcolor=macroblue]{hyperref}
\usepackage{varioref}
\usepackage{chngcntr}
\definecolor{darkgreen}{RGB}{30,175,88}
\definecolor{darkblue}{RGB}{30,118,175}
\definecolor{maroon}{rgb}{0.66,0,0}
\definecolor{darkgreen}{rgb}{0,0.69,0}
%Counters
\newtheorem{theorem}{Theorem}[section]
\newtheorem{proposition}{Proposition}
\newtheorem{lemma}{Lemma}
\newtheorem{corollary}{Corollary}
\newtheorem{assumption}{Assumption}
\newtheorem{axiom}{Axiom}
\newtheorem{case}{Case}
\newtheorem{claim}{Claim}
\newtheorem{condition}{Condition}
\newtheorem{definition}{Definition}
\newtheorem{example}{Example}
\newtheorem{notation}{Notation}
\newtheorem{remark}{Remark}
\hypersetup{ 	
pdfsubject = {The Career Rat Race},
pdftitle = {The Career Rat Race},
pdfauthor = {Pranay Gundam},
linkcolor= macroblue
}
\title{\textbf{The Career Rat Race}}
\author{Pranay Gundam}
%-----------------------
% Begin document
%-----------------------
\begin{document}
\maketitle


\section*{Introduction}

For some reason I always find myself gravitating towards these dynamic programming models; it just makes a lot of sense to me to treat these cyclical events as such.  In general, I like these sorts of questions where you can't get the answer by simply taking a first derivative and there are all these factors that act as push and pull forces on an individual's decision making process. The agents are also forced to think ahead since ones actions in previous walks of life has an effect on their actions in their current state and summarizing all of this in a bellman equation feels quite beautiful.

\section*{The Model}

I'm going to start this process in reverse compared to what I've typically done, by first introducing the bellman equation and then explaining all the notation for context. The agent/PhD econ applicant faces the following bellmanized decision problem $$v(t, x_{l}, x_{e}) = \max\left\{p  u(t) + \beta(1-p)v(t+1, x_{l}, x_{e}), w_{l} + \beta v(t+1, x_{l}+1, x_{e}), w_{e} + \beta v(t+1, x_{l}, x_{e}+1)\right\},$$ where $$u(t) = .$$

\noindent Utility is some function of total time discounted wages until retirement, personal delay preference (i.e. you dont like to delay going to grad school (maybe your wages are discounted more the larger t is)) and some other preference for being an academic. We can also tac on some other exogenous life experience and econ experience benefit as well.

\section*{Discussion}



\end{document}

