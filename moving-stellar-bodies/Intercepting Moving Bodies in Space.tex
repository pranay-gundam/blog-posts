%--------------------
% Packages
% -------------------
\documentclass[11pt,english]{article}
\usepackage{amsfonts}
\usepackage[left=2.5cm,top=2cm,right=2.5cm,bottom=3cm,bindingoffset=0cm]{geometry}
\usepackage{amsmath, amsthm, amssymb}
\usepackage{tikz}
\usetikzlibrary{calc}
\usetikzlibrary{decorations.pathreplacing,calligraphy}
\usepackage{fancyhdr}
%\usepackage{currfile}
\usepackage{nicefrac}
\usepackage{cite}
\usepackage{graphicx}
\usepackage{caption}
\usepackage{longtable}
\usepackage{rotating}
\usepackage{lscape}
\usepackage{booktabs}
\usepackage{float}
\usepackage{placeins}
\usepackage{setspace}
\usepackage[font=itshape]{quoting}
\onehalfspacing
\usepackage{mathrsfs}
\usepackage{tcolorbox}
\usepackage{xcolor}
\usepackage{subcaption}
\usepackage{float}
\usepackage[multiple]{footmisc}
\usepackage[T1]{fontenc}
\usepackage[sc]{mathpazo}
\usepackage{listings}
\usepackage{longtable}
\definecolor{cmured}{RGB}{175,30,45}
\definecolor{macroblue}{RGB}{56,108,176}
\usepackage[format=plain,
            labelfont=bf,
            textfont=]{caption}
\usepackage[colorlinks=true,citecolor=macroblue,linkcolor=macroblue,urlcolor=macroblue]{hyperref}
\usepackage{varioref}
\usepackage{chngcntr}

\definecolor{darkgreen}{RGB}{30,175,88}
\definecolor{darkblue}{RGB}{30,118,175}
\definecolor{maroon}{rgb}{0.66,0,0}
\definecolor{darkgreen}{rgb}{0,0.69,0}

%Counters
\newtheorem{theorem}{Theorem}[section] 
\newtheorem{proposition}{Proposition}
\newtheorem{lemma}{Lemma}
\newtheorem{corollary}{Corollary}
\newtheorem{assumption}{Assumption}
\newtheorem{axiom}{Axiom}
\newtheorem{case}{Case}
\newtheorem{claim}{Claim}
\newtheorem{condition}{Condition}
\newtheorem{definition}{Definition}
\newtheorem{example}{Example}
\newtheorem{notation}{Notation}
\newtheorem{remark}{Remark}



\hypersetup{ 	
pdfsubject = {},
pdftitle = {Intercepting Moving Bodies in Space},
pdfauthor = {Pranay Gundam},
linkcolor= macroblue
}


\title{\textbf{Intercepting Moving Bodies in Space}}
\author{Pranay Gundam}

%-----------------------
% Begin document
%-----------------------
\begin{document}

\maketitle

\subsection*{Introduction}

\noindent As I've been reading the \textit{Rememberance of Earth's Past} series by Lui Cixin one intellectual passion of mine has rekindled, Astrophysics. Now I will say, my experience in this subject is not as mathematically deep as it is in economics but I know the basics of kinematics and some differential equations as well which I think sets me up to model simple interactions between bodies in space. This leads me to talk about one question that I've always considered in the back of my mind and had a framework for solving but never really worked out the math for. I am curious about how space agencies choose how to target different celestial bodies, for example, when we send rovers to the Mars, I know it's not a simple matter of looking through a telescope at Mars and then setting the destination coordinates and heading there at full blast. One must consider the speed of light that creates a lagtime in our observations, the pattern with which the target celestial body travels such that we can aim to where we think it will be by the time we would get there, and the cost of materials it takes to travel at different speeds and for different lengths of time. There are undoubtably more factors to consider beyond the ones that I have listed but I wanted to start with a list such that I knew how to work with and was somewhat interesting.\\

\noindent The rest of this post shall be three exercises in increasing difficulty regarding how agents choose to get intercept celestial bodies in space where the difficulty lies in relative motion of the target body. I am completely aware that these are probably googleable problems but I wanted to work it out myself for fun.
\begin{enumerate}
	\item Stationary target 
	\item A target that moves 
	\item Further possible extensions and choices
\end{enumerate}

\subsection*{Stationary Target}

\noindent Consider our target to be located at $\mathbf{x_f^0}\in \mathbb{R}^3$ and the agent is located at $\mathbf{x_a^0}\in \mathbb{R}^3$. The goal of the agent is to move from their current location to the target but they face costs that depends on how long it takes to get to their destination and on how much they choose to accelerate throughout the duration of the trip. For now, we will assume that the agent chooses only one constant rate of acceleration throughout the entire journey. As such, the optimization problem that the agent faces is (I included some transformations to make the calculations a little bit easier) $$\min_{a_x,a_y,a_z,t}\,a_{cost}\lVert\mathbf{a}\rVert^2 + t_{cost}t$$ $$s.t. \quad \mathbf{x_f^0} = \mathbf{x_a^0} + \mathbf{v_0}t + \frac{1}{2}\mathbf{a}t^2$$

\noindent For now, I will also assume that the agent starts from a standstill (aka the agent has a 0 initial velocity). One important thing to note is that (and there might be a more clever way to solve it, just keep in mind I'm doing this blind with no resources) there are actually three boundary constraints, one for each of the three spatial dimensions. That being said, I will first solve the one dimensional case and then show how the solution is very similar in higer dimensions. For the one dimensional case we have that the lagrangian is
\begin{align*}
	\mathcal{L}(a,t,\lambda) = a_{cost}a^2 + t_{cost}t &- \lambda(x_{a}^0 - x_{f}^0 + 0.5at^2).
\end{align*}
The corresponding first order conditions are then
\begin{align*}
a:& 2a_{cost}a - 0.5\lambda t^2 = 0\\
t:& t_{cost} - \lambda at = 0\\
\lambda:& x_{a}^0 - x_{f}^0 + 0.5at^2 = 0\\
\end{align*}
The third FOC implies that $a = \frac{2(x_f^0 - x_a^0)}{t^2}$ which we can plug into the first two conditions to get the system
\begin{align*}
\frac{4a_{cost}(x_f^0 - x_a^0)}{t^2} - 0.5\lambda t^2 &= 0  \\
t_{cost} - \frac{2\lambda (x_f^0 - x_a^0)}{t}  &= 0 \implies \lambda = \frac{t_{cost}t}{2(x_f^0 - x_a^0)}\\
\end{align*} and then 
\begin{align*}
& \frac{4a_{cost}(x_f^0 - x_a^0)}{t^2} - 0.5\lambda t^2 = 0  \\
\implies & \frac{4a_{cost}(x_f^0 - x_a^0)}{t^2} -  \frac{t_{cost}t^3}{4(x_f^0 - x_a^0)} = 0  \\
\implies & t^5 = \frac{16a_{cost}(x_f^0 - x_a^0)^2}{t_{cost}}\\
\implies & \pmb{t = \left(\frac{16a_{cost}(x_f^0 - x_a^0)^2}{t_{cost}}\right)^{1/5}}\\
\implies & \pmb{a = \left(\frac{t_{cost}^2(x_f^0 - x_a^0)}{8a_{cost}^2}\right)^{1/5}}.
\end{align*}
Intuitively, this answer aligned with very basic dynamics that we would expect, both the optimal time an acceleration are inversly related to their respected costs and the ratio between the two costs is important in deciding what levels of either variable is optimal. Now for the three dimensional case 
\begin{align*}
	\mathcal{L}(a_x,a_y,a_z,\lambda_x,\lambda_y,\lambda_z, t) = a_{cost}\left[a_x^2 + a_y^2 + a_z^2\right] + t_{cost}t &- \lambda_x(x_{a,x}^0 - x_{f,x}^0 + 0.5a_xt^2)\\ &- \lambda_y(x_{a,y}^0 - x_{f,y}^0 + 0.5a_yt^2)\\ &- \lambda_z(x_{a,z}^0 - x_{f,z}^0 + 0.5a_zt^2).
\end{align*}

The corresponding first order conditions are then
\begin{align*}
a_x&: 2a_{cost}a_x - 0.5\lambda_xt^2 = 0\\
a_y&: 2a_{cost}a_y - 0.5\lambda_yt^2 = 0\\
a_z&: 2a_{cost}a_z - 0.5\lambda_zt^2 = 0\\
t&: t_{cost} - \lambda_x a_xt - \lambda_y a_yt - \lambda_z a_zt =0\\ 
\lambda_x&: x_{a,x}^0 - x_{f,x}^0 + 0.5a_xt^2 = 0\\
\lambda_y&: x_{a,y}^0 - x_{f,y}^0 + 0.5a_yt^2 = 0\\
\lambda_z&: x_{a,z}^0 - x_{f,z}^0 + 0.5a_zt^2 = 0
\end{align*}

So we can then continue from the FOC conditions
\begin{align*}
x_{a,x}^0 - x_{f,x}^0 + 0.5a_xt^2 &= 0 \implies  a_x = \frac{2( x_{f,x}^0 - x_{a,x}^0)}{t^2}, a_y = \frac{2( x_{f,y}^0 - x_{a,y}^0)}{t^2}, a_z = \frac{2( x_{f,z}^0 - x_{a,z}^0)}{t^2}\\	
2a_{cost}a_x - 0.5\lambda_xt^2 &= 0 \implies \lambda_x = \frac{4a_{cost}a_x}{t^2}, \lambda_y = \frac{4a_{cost}a_y}{t^2}, \lambda_z = \frac{4a_{cost}a_z}{t^2}\\
\end{align*}
Combining the two expressions above we then get $$\lambda_x = \frac{8a_{cost}( x_{f,x}^0 - x_{a,x}^0)}{t^4}, \lambda_y = \frac{8a_{cost}( x_{f,y}^0 - x_{a,y}^0)}{t^4}, \lambda_z = \frac{8a_{cost}( x_{f,z}^0 - x_{a,z}^0)}{t^4}.$$ After substituting this into the third FOC and some more algebra we get that \begin{align*}
\pmb{t} &\pmb{= \left(\frac{16a_{cost}}{t_{cost}}\lVert x_{f}^0 - x_a^0 \rVert^2\right)^{1/5}}\\
\pmb{a_x} &\pmb{= \left(\frac{t_{cost}^2}{8a_{cost}^2}\lVert x_{f}^0 - x_a^0 \rVert^{-4}\right)^{1/5}(x_{f,x}^0 - x_{a,x})}\\
\pmb{a_y} &\pmb{= \left(\frac{t_{cost}^2}{8a_{cost}^2}\lVert x_{f}^0 - x_a^0 \rVert^{-4}\right)^{1/5}(x_{f,y}^0 - x_{a,y})}\\
\pmb{a_z} &\pmb{= \left(\frac{t_{cost}^2}{8a_{cost}^2}\lVert x_{f}^0 - x_a^0 \rVert^{-4}\right)^{1/5}(x_{f,z}^0 - x_{a,z})}
\end{align*}
\noindent If for some reason we wanted to consider our answer in $n$ many dimensions for any $n\in \mathbb{N}$ we can see how the expression would be the same. One important detail to note is how each of acceleration choices in each dimension are dependent on some level of a ratio between the distance in that specific dimension from the starting point to the target and the total distance in each of the dimensions accumulated.

\subsection*{Target with non-zero velocity and acceleration}

I've worked out the algebra for this to an extent and it is not fun. I'll note down a little bit of it here but I don't think I can come to a nice closed form expression as is above. To begin, we consider practically the same problem setup as before but now the boundary constraint is changed so that the target has values for initial velocity and acceleration. Specifically, we are faced with the optimization problem  $$\min_{\mathbf{a},t}\,a_{cost}\lVert\mathbf{a}\rVert^2 + t_{cost}t$$ $$s.t. \quad \mathbf{x_f^0} + \mathbf{v^0_f}t + \frac{1}{2}\mathbf{a_f}t^2= \mathbf{x_a^0} + \mathbf{v^0_a}t + \frac{1}{2}\mathbf{a_a}t^2$$

\noindent The three dimensional case is a bit much to take on initially so with the fact that the solutions in each of the dimensions are pretty parallel I'll only consider the one dimensional case. In this case we have that the lagrangian is
\begin{align*}
	\mathcal{L}(a,t,\lambda) = a_{cost}a^2 + t_{cost}t &- \lambda((x_{a}^0 - x_{f}^0) + (v_a^0-v_f^0)t + 0.5(a_a - a_f)t^2).
\end{align*}
The corresponding first order conditions are then
\begin{align*}
a:& 2a_{cost}a - 0.5\lambda t^2 = 0\\
t:& t_{cost} - \lambda((v_a^0-v_f^0) + (a_a - a_f)t) = 0\\
\lambda:& (x_{a}^0 - x_{f}^0) + (v_a^0-v_f^0)t + 0.5(a_a - a_f)t^2 = 0\\
\end{align*}

The furthest that I can workout the first order conditions          


\noindent Before we head off into the next discussion, it is important to note that when there are no outside forces that act upon the agent we can actually shift our frame of reference and reduce a problem of n-dimensions into just a problem of one dimension (in fact even if there are forced acting upon the agent as long as they are parallel to the hyperplane that passes through the agent's starting and optimal ending points, but I think there is some additional nuance here with the optimal starting and ending points being dependent on the exogenous forces).

\subsection*{Further extensions of the problem}

So far I've been pitching the choices as choosing acceleration and time, but there are quite a few expressions in physics that can relate this position function. For example, we could model how the mass of a space ship increases as the length of a trip increases (for example a ship will require more resources and fuel on board for longer trips with a constant rate of acceleration) by substituting $a = F/m(t)$ where mass is written as a function of time and instead of choosing $a$ in the optimization problem, we choose $F$. We could even take further steps to expand $F$ with the energy equation $E = Fd$ and if then we were NASA we could easily know how much fuel exaclty we need to pack (whether chemical or \textit{fission}).





\end{document}